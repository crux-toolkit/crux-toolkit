Reviewer Comments to Author:

Reviewer: 1
Comments to the Author
In their new manuscript, Diament et al consider a small set of the
possibilities for improving MS/MS sampling with the addition of
real-time identification capabilities.  Their report, however, is
hampered by its evaluation approach, which is rooted in a simulation
that produces results differing substantially from real-world
performance.

The maximum exclusion list size of 50 cited by this article dates from
LCQ-era mass spectrometers.  Typically, LTQ and later instruments
featured an exclusion list size of 500.

RE-DO EXPERIMENTS WITH EXCLUSION LIST OF 500.

The data sets employed in this manuscript were previously published,
but the authors really should include more information about them in
this manuscript.  At a bare minimum, stating that they include MS
scans from an Orbitrap is rather important for understanding the new
analysis.

ADD THIS INFO (TRIVIAL).

Purists might object to the first paragraph of Results, which suggests
that all candidate peptides are subjected to XCorr.  At least in the
original Sequest, only the five hundred best preliminary scoring
peptides are subjected to XCorr.

TRIVIAL FIX.

The data in Figure 5 cast significant doubt on the ability of the
simulator to predict real-world mass spectrometer performance.  The
authors need to buttress the reader's faith that the simulator is
providing useful information in understanding instrument performance.
Until this discrepancy is explained better, all the conclusions of
this article are in doubt.

RE-STATE EXPLANATIONS FROM DISCUSSION.

In the authors' description of the variable duration exclusion list,
it would seem that determining whether or not a spectrum has been
acquired recently does not require identification but rather merely a
determination of whether or not the same fragments have recently been
collected in another spectrum.  This is feasible through
spectrum-spectrum comparison without identification.

TRUE, BUT THIS IS A SEPARATE IDEA.

The following paragraph would appear to reach two opposite predictions
from the same scenario: The second peak-picking method that we
investigated is based on the hypothesis that if a peak is not
confidently identified, this may be due to noise, and a second attempt
may enable a more confident identification on the second try (probably
with the same result as before).  On the other hand, if a peak yields
a poor spectrum, then there is no point in retrying, as a new attempt
is likely to give the same difficulties.

RE-WORD TO CLARIFY.

At present, the Weibull-score Triage section describes a strategy that
seems counterintuitive, at best, to this reader.  I'm unsurprised that
they saw poor results in applying it.  At the moment this technique
seems ill-considered, and I don't see any lessons from this section to
be taken away by readers.

ARGUE FOR WHY IT'S NOT ILL-CONSIDERED.

The charge-state exclusion rule seems reasonable, but again I note
that this rule can be implemented without knowledge of peptide
identity.  The instrument control software can already determine the
charge of precursors from isotope distributions, and so it can
determine where other charges for a precursor can be found without an
identification.

BUT THE PEPTIDE ID MAKES THE EXCLUSION MORE PRECISE.

Are the authors sure that this type of exclusion is
not already practiced?

NO.

Given that we can discriminate identifications
for +2s more effectively than for other charge states, it would seem
that preferring this charge over all others, even when they are more
intense, would be advantageous.

TRUE.

If the authors are plotting distinct sequence counts (rather than
identified MS/MS counts) on the y axis of Figure 6, this might be
important to emphasize, especially in the context of the charge state
exclusion rule.

FIX THIS.

The authors may want to discuss how ions were selected as "most
intense" for DDA.  Are they working from centroids or from profile MS
scans?  Are they including isotopes in their assessment of intensity?
Under what rules does Thermo's software appear to be assessing peak
intensity?

FIX THIS.

If the authors are looking for very large numbers of replicates for a
sample, they might consult with PNNL for the many LC-MS/MS files they
have generated from Shewanella.  Another possibility would be the many
LC-MS/MS shots of yeast collected by the CPTAC network.  If a simpler
mixture would be more tractable, a QC set for a simple protein mixture
such as the ISB 18 would be useful.

DON'T DO THIS.


Reviewer: 2
Comments to the Author
The key premise of this manuscript is that having the ability for
real-time identification of tandem mass spectra should allow one to
redesign LC/MS precursor selection strategies to further increase
the number of identified MS/MS spectra. To assess the possible
feasibility of such strategies, the authors created a precursor
selection simulator based on pooled data from 11 real LC/MS/MS runs
and used it to show that simple precursor selection strategies
performed substantially worse (i.e., about 18% worse) than current
m/z and retention-time exclusion window strategies.

Overall this manuscript suffers from multiple serious issues.

The first such issue is that the core hypothesis being tested is
not clearly defined anywhere in the manuscript. While the overall
premise that real-time identifications should allow one to improve
precursor selection strategies, this concept is never explicitly
translated to a testable hypothesis. The closest to such a
formulation is given only on page 12 at the start of the
"Weibull-score Triage" section where it is indicated that
peptide-spectrum match (PSM) scores from real-time identification
would be used to decide on the utility of precursor reacquisition
in the hope of obtaining higher quality MS/MS spectra on
reacquisition. Overall it remains unclear throughout the manuscript
exactly how the authors propose that real-time identification
should improve MS/MS identification rates other than through
reacquisition with hopefully slightly higher signal to noise
(though it was not assessed how often reacquisition did result in
higher signal to noise nor how much this matters for
identification).

EASY TO FIX THIS.

The second issue is that a manuscript aiming to report negative
results needs to do a much more extensive and detailed presentation
of the tested methods than manuscripts presenting positive results.
In the latter case authors only need to report what worked well but
in negative-results manuscripts the main contribution is that
'something cannot be done in a certain way' so it is key to clearly
define the 'something' being tested and to clearly describe the
'certain way' and all the data used to assess it. The lack of such
detailed assessment (or its reporting in supplementary materials)
is prevalent throughout the paper but can also be illustrated by
following up on the previous paragraph's comments on the
Weibull-score triage strategy. The hypothesis underlying this
strategy is that Weibull-score is a good indicator of whether an
MS/MS spectrum will be more 'identifiable' upon reacquisition due
to increased signal-to-noise ratio. However, this is not shown even
in principle (e.g., by showing a plot of isolated ion intensity for
each MS/MS spectrum vs its Weibull-score for accepted/rejected
PSMs) and is not compared with the trivial strategy of just using
MS/MS ion intensity nor compared with multiple competing strategies
that assess spectrum quality without attempting any identification.
A similar argument can be made for the variable duration exclusion
list strategy - it would probably be much more efficient to adjust
the exclusion time based on LC/MS retention time traces than to
have to 'invest' another MS/MS event just to determine whether the
same peptide is still there (which could be trivially assessed
using MS/MS spectrum matching, no identification is required
anyway) but these simpler strategies were not compared to the
real-time identification strategy.

Third, in addition to missing comparisons with alternative simpler
strategies, the manuscript provides almost no details about exactly
what methods were tested. For example, the simulator is said to
have 'several interesting and uninteresting parameters' but these
are not shown in detail anywhere in the manuscript.

THIS IS A GOOD POINT, AND EASY TO FIX.

In a negative
results manuscript, these should have been thoroughly described and
separately assessed against for their impact on precursor selection
strategies. This is especially relevant since the proposed
simulator was not even able to reproduce the behavior of current
dynamic exclusion strategies - such lack of details makes it
impossible to determine which parameters may have been missing or
set to inappropriate values. There were also multiple claims that
exclusion lists are unstable but no detailed examples or
statistical data was provided to support this assessment. For
example, how many and what fraction of "skipped peaks could not be
'explained away' in this fashion" as claimed in page 10? What were
the characteristics of those peaks that caused the exceptional
behavior? What the most striking detailed example of how small
differences in exclusion parameters caused dramatic instability
about precursor selection later in the LC/MS run?

GOOD POINTS.

Fourth, one of the justifications for the suboptimal performance of
the proposed approach was that it often selected peaks for which
MS/MS spectra were not available to the simulator even after
merging 11 replicate LC/MS runs. These include 5 PAnDA runs and
thus are expected to include roughly six times the number of
precursors that would be acquired in a single LC/MS run with
dynamic exclusion. As such, it is strange that the proposed
strategies would be selecting peaks that did not "look interesting
to the instrument" even after 5 PAnDA runs. What are these "odd"
precursors that the instrument software chooses to ignore so
consistently but are picked by the proposed strategies? Are there
any features of these precursors that would be indicative of their
identifiability? For example, do these "odd" features contain
unusual isotope patterns or correspond to co-eluting peptides?

Finally, even if the proposed strategies had improved the MS/MS
identification rate and absolute numbers, the manuscript would
still have a steep hill to climb to show that the trade-offs of
real-time acquisition using Tide+q-ranker also surpass the
identification performance of standard dynamic exclusion coupled
with standard identification strategies such as Mascot+Percolator.
Ultimately this is the bar that must be passed for real-time
acquisition to gain any acceptance and this comparison was
completely ignored in this manuscript.
